    %Type setting your answer here.

\documentclass{article}
\usepackage{amsmath,amssymb,amsthm}
\usepackage[colorlinks = true,
            linkcolor = blue,
            urlcolor  = blue,
            citecolor = blue,
            anchorcolor = blue]{hyperref}

\newcounter{question}
\newenvironment{question}[1][]{\refstepcounter{question}\par\medskip
   \noindent\textbf{\thequestion. #1} \rmfamily\begin{bfseries}}{\end{bfseries}\medskip}
   
\newcounter{subquestion}[question]
\newenvironment{subquestion}[1][]{\refstepcounter{subquestion}\par\medskip
   \noindent\textbf{\thequestion.\thesubquestion #1} \rmfamily\begin{bfseries}}{\end{bfseries}\medskip}

\title{CSC 565 2020 Fall Homework 2}
%replace name and unityID for all group members.
\author{    
    name1, unityID1 \\
    \and
    name2, unityID2 
}

\begin{document}

  \maketitle
  
  \noindent You can create a latex project of this homework through this \href{https://www.overleaf.com/docs?snip_uri=https://donsheehy.github.io/graphtheoryhomework/hw2.tex}{link}.

  \section*{Cartesian Products}
    Let $G$, $H$ be graphs. The Cartesian product between $G$ and $H$ can be defined as:
    %TODO: need update for clarity.
    \begin{alignat*}{2}
        G \Box H := (& V_G \times V_H, &&~\\
        &\{(ab, cd)~|~ & (a=c~&\text{and}~(b, d)\in E_H)\\
        &~ & &\text{or} \\
        &~ & (b=d~&\text{and}~(a, c)\in E_G), \\
        &~ & a,c &\in V_G, \\
        &~ & b, d, &\in V_H\})
    \end{alignat*}
    \begin{question}
        Give a function $f_V: V_{G \Box H} \rightarrow V_{H \Box G}$ that induces an isomorphism. No need to prove it is an isomorphism. 
    \end{question}
    \\\\
    \textit{Answer:}
    \\\\
    %%%%%%%%%%%%%%%%%%%%%%%%%%%%%%%%%%%%%%%%%%%%%%%%%%%%%%%%%
    %Type setting your answer here.
    
    %%%%%%%%%%%%%%%%%%%%%%%%%%%%%%%%%%%%%%%%%%%%%%%%%%%%%%%%%
    \begin{question}
        Prove that for all $i, j \geq 3$, $C_j \Box P_j$ is planar. \\
        Hint: You can try to do it by providing an embedding for the general case.
    \end{question}
    \\\\
    \textit{Answer:}
    \\\\
    %%%%%%%%%%%%%%%%%%%%%%%%%%%%%%%%%%%%%%%%%%%%%%%%%%%%%%%%%
    %Type setting your answer here.
    
    %%%%%%%%%%%%%%%%%%%%%%%%%%%%%%%%%%%%%%%%%%%%%%%%%%%%%%%%%
    \begin{question}
        Prove $C_3 \Box C_3$ is \underline{NOT} planar by completing the following steps. It is a fact that $C_3 \Box C_3$ is 3-connected. Think about this and convince yourself of this first.
    \end{question}
    \begin{subquestion}
        Suppose it is planar. How many faces should it have according to Euler's formula?
    \end{subquestion}
    \\\\
    \textit{Answer:}
    \\\\
    %%%%%%%%%%%%%%%%%%%%%%%%%%%%%%%%%%%%%%%%%%%%%%%%%%%%%%%%%
    %Type setting your answer here.
    
    %%%%%%%%%%%%%%%%%%%%%%%%%%%%%%%%%%%%%%%%%%%%%%%%%%%%%%%%%
    \begin{subquestion}
        How many non-separating induced cycles are there? You may assume without proving it that all such cycles have length 3 or 4.
    \end{subquestion}
    \\\\
    \textit{Answer:}
    \\\\
    %%%%%%%%%%%%%%%%%%%%%%%%%%%%%%%%%%%%%%%%%%%%%%%%%%%%%%%%%
    %Type setting your answer here.
    
    %%%%%%%%%%%%%%%%%%%%%%%%%%%%%%%%%%%%%%%%%%%%%%%%%%%%%%%%%
    \begin{subquestion}
        Complete the proof.
    \end{subquestion}
    \\\\
    \textit{Answer:}
    \\\\
    %%%%%%%%%%%%%%%%%%%%%%%%%%%%%%%%%%%%%%%%%%%%%%%%%%%%%%%%%
    %Type setting your answer here.
    
    %%%%%%%%%%%%%%%%%%%%%%%%%%%%%%%%%%%%%%%%%%%%%%%%%%%%%%%%%

%%%%%%%%%%%%%%%%%%%%%%%%%%%%%%%%%%%%%%%%%%%%%%%%%%%%%%%%%%%%%%%%%%%%%%%%%%%%%%%%%%%%%
\newpage
  \section*{Outerplanar Graphs}
    \begin{question}
        Let $G$ be an outerplanar graph with $n$ vertices and as many edges as possible but subject to the constraint that $G$ is not 2-connected. How many edges does $G$ have? Prove it.
    \end{question}
    \\\\
    \textit{Answer:}
    \\\\
    %%%%%%%%%%%%%%%%%%%%%%%%%%%%%%%%%%%%%%%%%%%%%%%%%%%%%%%%%
    %Type setting your answer here.
    
    %%%%%%%%%%%%%%%%%%%%%%%%%%%%%%%%%%%%%%%%%%%%%%%%%%%%%%%%%
    \begin{question}
        Let $G$ be the graph:
        \begin{align*}
            G = ([n], \{(i, j)~|~(j-i)~\%~n = 1 \lor (j-i)~\%~n = 2\}), n \geq 4
        \end{align*}
        $\%$ is the modulo operator. Prove that $G$ is \underline{NOT} outerplanar.
    \end{question}
    \\\\
    \textit{Answer:}
    \\\\
    %%%%%%%%%%%%%%%%%%%%%%%%%%%%%%%%%%%%%%%%%%%%%%%%%%%%%%%%%
    %Type setting your answer here.
    
    %%%%%%%%%%%%%%%%%%%%%%%%%%%%%%%%%%%%%%%%%%%%%%%%%%%%%%%%%
    
%%%%%%%%%%%%%%%%%%%%%%%%%%%%%%%%%%%%%%%%%%%%%%%%%%%%%%%%%%%%%%%%%%%%%%%%%%%%%%%%%%%%%
\newpage
  \section*{Extra Questions}
    These questions are for your interest and practice. It's recommended to think about them. They will not be graded.
    \begin{question}
        Prove that every outerplanar graph is 3-colorable.
    \end{question}
    \begin{question}
        Prove that no outerplanar graph is 3-connected.
    \end{question}

\end{document}