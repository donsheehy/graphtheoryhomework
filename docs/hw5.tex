%Type setting your answer here.

\documentclass{article}
\usepackage{amsmath,amssymb,amsthm}
\usepackage[colorlinks = true,
        linkcolor = blue,
        urlcolor  = blue,
        citecolor = blue,
        anchorcolor = blue]{hyperref}

\newcounter{question}
\newenvironment{question}[1][]{\refstepcounter{question}\par\medskip
\noindent\textbf{\thequestion. #1} \rmfamily\begin{bfseries}}{\end{bfseries}\medskip}

\newcounter{subquestion}[question]
\newenvironment{subquestion}[1][]{\refstepcounter{subquestion}\par\medskip
\noindent\textbf{\thequestion.\thesubquestion #1} \rmfamily\begin{bfseries}}{\end{bfseries}\medskip}

\title{CSC 565 2020 Fall Homework 5}
%replace name and unityID for all group members.
\author{
name1, unityID1 \\
\and
name2, unityID2
}

\begin{document}

\maketitle

\noindent You can create a latex project of this homework through this \href{https://www.overleaf.com/docs?snip_uri=https://donsheehy.github.io/graphtheoryhomework/hw5.tex}{link}.

\section*{Boundary Matrices}

\begin{question}
  Write out the boundary matrix for the following graphs:
  \begin{enumerate}
    \item $K_3$
    \item $P_3$
    \item $K_2\times K_2$
  \end{enumerate}
\end{question}

\textit{Answer:}
\\\\
%%%%%%%%%%%%%%%%%%%%%%%%%%%%%%%%%%%%%%%%%%%%%%%%%%%%%%%%%
%Typeset your answer here.

%%%%%%%%%%%%%%%%%%%%%%%%%%%%%%%%%%%%%%%%%%%%%%%%%%%%%%%%%

\begin{question}
  Let $T$ be a tree with $5$ vertices.  Let $\partial$ be the boundary matrix for $T$.
  Let $A$ be the $4\times 4$ matrix formed by deleting one row of $\partial$.
  What are the possible values of $\det(A)$?
\\\\
  Hint: Try to find an ordering on vertices such that $A$ is an upper-triangular matrix. 
\end{question}
\\\\
\textit{Answer:}
\\\\
%%%%%%%%%%%%%%%%%%%%%%%%%%%%%%%%%%%%%%%%%%%%%%%%%%%%%%%%%
%Typeset your answer here.

%%%%%%%%%%%%%%%%%%%%%%%%%%%%%%%%%%%%%%%%%%%%%%%%%%%%%%%%%

\emph{Challenge to think about: What if $T$ were not a tree?}

\section*{The Adjacency Matrix}

Remember $M^k = \overbrace{M* ... * M}^{k}$, where $M$ is a square matrix and $*$ is matrix multiplication.
\\\\
Here is an interesting fact about the adjacency matrix $A_G$ of a graph $G$.
For any $k$, the entries $[A_G^k]_{ij}$ are the number of distinct walks from $i$ to $j$ of length $k$ in $G$.
If you haven't seen this before, you might want to work it out by expanding the matrix multiplication and applying induction.
This fact has the following interesting consequences for you to prove.

\begin{question}
  Prove that the trace of $A_G^3$ is equal to six times the number of $K_3$ subgraphs in $G$.
\end{question}
\\\\
\textit{Answer:}
\\\\
%%%%%%%%%%%%%%%%%%%%%%%%%%%%%%%%%%%%%%%%%%%%%%%%%%%%%%%%%
%Type setting your answer here.

%%%%%%%%%%%%%%%%%%%%%%%%%%%%%%%%%%%%%%%%%%%%%%%%%%%%%%%%%

\begin{question}
  Let $M = A_G^2 - L_G$, where $L_G$ is the Laplacian of $G$.
  Prove that $M_{ij}$ is the number of paths (not walks) from $v_i$ to $v_j$ of length $1$ or $2$.
\end{question}
\\\\
\textit{Answer:}
\\\\
%%%%%%%%%%%%%%%%%%%%%%%%%%%%%%%%%%%%%%%%%%%%%%%%%%%%%%%%%
%Typeset your answer here.

%%%%%%%%%%%%%%%%%%%%%%%%%%%%%%%%%%%%%%%%%%%%%%%%%%%%%%%%%


\end{document}
